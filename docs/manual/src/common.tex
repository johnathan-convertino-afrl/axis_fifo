\begin{titlepage}
  \begin{center}

  {\Huge AXIS\_FIFO}

  \vspace{25mm}

  \includegraphics[width=0.90\textwidth,height=\textheight,keepaspectratio]{img/AFRL.png}

  \vspace{25mm}

  \today

  \vspace{15mm}

  {\Large Jay Convertino}

  \end{center}
\end{titlepage}

\tableofcontents

\newpage

\section{Usage}

\subsection{Introduction}

\par
Standard AXIS FIFO with multiple options. The FIFO uses a AXIS interface for data in and out.
It also emulates the Xilinx AXIS FIFO bugs and all. This is NOT dependent on Xilinx FPGA's and
can be used on any FPGA supporting the Verilog block ram style primitive.

\subsection{Dependencies}

\par
The following are the dependencies of the cores.

\begin{itemize}
  \item fusesoc 2.X
  \item iverilog (simulation)
  \item cocotb (simulation)
\end{itemize}

\input{src/fusesoc/depend_fusesoc_info.tex}

\subsection{In a Project}
\par
Simply use this core between a sink and source devices. This buffer data from one bus to another. Check the code to see if others will work correctly.

\section{Architecture}

\par
This AXIS FIFO is made for two modules. They are the FIFO, and AXIS FIFO control. The combination of these two provide the AXIS FIFO module.
Having it made this way allows for future modules to be customized and brought in to change the FIFO's behavior. The current modules emulate the Xilinx
AXIS FIFO IP core availble in Vivado 2018 and up.

\par
AXIS FIFO control is the heart of the core when it comes to how it responds. The logic in the core is designed to emulate the Xilinx AXIS FIFO IP.

Please see \ref{Module Documentation} for more information.

\section{Building}

\par
The AXIS FIFO core is written in Verilog 2001. They should synthesize in any modern FPGA software. The core comes as a fusesoc packaged core and can be
included in any other core. Be sure to make sure you have meet the dependencies listed in the previous section.

\subsection{fusesoc}
\par
Fusesoc is a system for building FPGA software without relying on the internal project management of the tool. Avoiding vendor lock in to Vivado or Quartus.
These cores, when included in a project, can be easily integrated and targets created based upon the end developer needs. The core by itself is not a part of
a system and should be integrated into a fusesoc based system. Simulations are setup to use fusesoc and are a part of its targets.

\subsection{Source Files}

\subsubsection{fusesoc\_info File List}
\begin{itemize}
\item src
	\begin{itemize}
	\item[$\space$] Type: verilogSource
	\item src/axis\_fifo.v
	\item src/axis\_fifo\_ctrl.v
	\end{itemize}
\item tb
	\begin{itemize}
	\item {'tb/tb\_axis.v': {'file\_type': 'verilogSource'}}
	\end{itemize}
\item ut
	\begin{itemize}
	\item {'ut/cocotb\_axis\_verification.py': {'file\_type': 'user', 'copyto': '.'}}
	\end{itemize}
\end{itemize}


\subsection{Targets}

\input{src/fusesoc/targets_fusesoc_info.tex}

\subsection{Directory Guide}

\par
Below highlights important folders from the root of the directory.

\begin{enumerate}
  \item \textbf{docs} Contains all documentation related to this project.
    \begin{itemize}
      \item \textbf{manual} Contains user manual and github page that are generated from the latex sources.
    \end{itemize}
  \item \textbf{src} Contains source files for the core
  \item \textbf{tb} Contains test bench files for iverilog and cocotb
    \begin{itemize}
      \item \textbf{cocotb} testbench files
    \end{itemize}
\end{enumerate}

\newpage

\section{Simulation}
\par
There are a few different simulations that can be run for this core.

\subsection{iverilog}
\par
iverilog is used for simple test benches for quick verification, visually, of the core.

\subsection{cocotb}
\par
Future simulations will use cocotb. This feature is not yet implemented.

\newpage

\section{Module Documentation} \label{Module Documentation}

\par
There is a single async module for this core.

\begin{itemize}
\item \textbf{AXIS\_FIFO} will buffer data from input to output.\\
\item \textbf{AXIS\_FIFO\_CONTROL} emulates the Xilinx FIFO IP interface and its behavior.\\
\end{itemize}
The next sections document the module in great detail.

